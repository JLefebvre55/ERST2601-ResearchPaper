\documentclass{report}
\usepackage{setspace} % Setting line spacing
\usepackage[normalem]{ulem} % Underline
\usepackage{caption} % Captioning figures
\usepackage{subcaption} % Subfigures
\usepackage{geometry} % Page layout
\usepackage{multicol} % Columned pages
\usepackage{array,etoolbox}
\usepackage{fancyhdr}
\usepackage{enumitem}
\usepackage[table]{xcolor}
\usepackage[toc,page]{appendix}
\usepackage{titlesec} % Section formatting

\usepackage[backend=biber,style=apa,citestyle=authoryear]{biblatex}
\DeclareLanguageMapping{english}{english-apa}
\DeclareFieldFormat{journaltitle}{\textit{#1}}
\DeclareFieldFormat[article]{volume}{\textit{#1}}
\DeclareFieldFormat[misc]{title}{\textit{#1}}
\DeclareFieldFormat[inbook]{booktitle}{\textit{#1}}
\DeclareFieldFormat[book]{title}{\textit{#1}}
\renewcommand*{\nameyeardelim}{\addcomma\space}
\DeclareDelimFormat[parencite]{finalnamedelim}{\addspace\&\space} % Use `&' instead of `and' in citations
\addbibresource{references.bib}

\titleformat{\section}{\normalfont\fontsize{12}{15}\bfseries}{\thesection}{1em}{}
\titleformat{\subsection}{\normalfont\fontsize{12}{15}\bfseries}{\thesubsection}{1em}{}

\AtBeginEnvironment{quote}{\par\onehalfspacing\small} % Block quotes

% Page layout (margins, size, line spacing)
\geometry{letterpaper, left=1in, right=1in, bottom=1in, top=1in}
\setstretch{2}

% Headers
\pagestyle{fancy}
\lhead{ERST2601 Final Research}
\rhead{Jayden Lefebvre}
\setlength{\headheight}{16pt}

\begin{document}

\begin{titlepage}
    \begin{center}
        \vspace*{1.2cm}

        \textbf{Addressing Food Insecurity in Yellowknife Dene First Nation: Examining the Synthesis of Western Agricultural Sciences with Indigenous Food Sovereignty Practices in Securing Endangered Traditional Foodways}

        \vspace{2cm}

        Jayden Lefebvre\\

        \vspace{5cm}
        
        Trent University\\
        ERST 2601Y 2024WI\\
        Dr. Dan Longboat\\

        \vfill

        March 17th, 2025
        
    \end{center}
\end{titlepage}

\thispagestyle{plain}
\tableofcontents

\clearpage

\section{Introduction}

\subsection{Position}

\hspace{24pt} I will begin by explaining my position as it pertains to the subject of this research.
I was born at the turn of the 21st century in Oshawa, Ontario, Canada on the land within the traditional territory of the Michi Saagiig Nishnaabeg, where I continue to live, work, and study in and around Peterborough, known traditionally as Nogojiwanong.
I was raised in a white Christian working-middle-class family, with two gainfully-employed and able-bodied parents.
My father, who is of mixed French and Irish immigrant heritage, is a skilled tradesman and entrepreneur who who taught me objectivity, problem-solving skills, and Western science.
My mother, who is of mixed English and Scottish settler heritage, is a unionized secondary school teacher of English and History who instilled in me the importance of compassion, pacifism, and humility.
Both of my parents exemplified self-reliance and independence, and raised me to value the transformative processes of education and self-improvement, and to not be ashamed of who I am.

\hspace{24pt} I have grown up with certain privileges attributed to the exploitative and violent processes of colonization.
I have never experienced and will likely never experience food insecurity, homelessness, or discrimination.
I will never understand what it means to be Indigenous, or to be the subject of a victimizing and exclusionary society.
As such I recognize that the things I will attempt to talk about here are more than they appear to my eyes.
Despite this fact, I will do my utmost to approach this research with humility and respect, and to acknowledge the limitations of my perspective as best I can.

\clearpage

\subsection{Background}

\hspace{24pt} The Dene First Nation have inhabited Yellowknife, Northwest Territories, Canada since time immemorial, and have inherited a complex and evolving traditional knowledge system of the land and its ecosystems \parencite{lorecapturingtraditional}.
The food security of the peoples of the Yellowknife Dene First Nation (YKDFN) largely relies on the gathering of wild food sources for sustenance;
Boyd et al. evaluated through structured interviews conducted in 2005 that ``subsistence harvesting continues to play an important role [alongside] participation in some form of wage-earning labour'' \parencite[268]{socialculturalcapital}.
Despite this, the YKDFN have faced significant challenges in maintaining food security due to the high cost of food imports, unemployment and poverty, and environmental, economic, and policy barriers to local food production \parencite{resilientcommunities};
Retired Chief of the YKDFN, Jonas Sangris, stated during a 2025 interview that ``now, it's different, I think it's hard; everything costs so much now'' [47:20] \parencite{jonassangris}.

\hspace{24pt} Food insecurity is a complex phenomenon that has disproportionately and severely affected the standard of living of Indigenous Peoples in Canada.
Statistics Canada and Health Canada define food insecurity as ``the inability to acquire or consume an adequate diet quality or sufficient quantity of food in socially acceptable ways, or the uncertainty that one will be able to so'' \parencite[para. 2]{statscanfoodinsecurity}.
A report generated by an expert panel on behalf of the Council of Canadian Academies in 2014 published the following finding:

\begin{quote}
  Rates of food insecurity were more pronounced among adult respondents in the 16 communities of the Dene Nation of the NWT: from 2008 to 2010, more than 90\% of the 824 respondents indicated that in the 12 months before the survey, they or other adults in their household either had cut the size of their meals or skipped meals, were hungry but did not eat, or ate less than they felt they should — in all cases, due to a lack of money for food. \parencite[42]{aboriginalfoodsecurity}
\end{quote} % TODO: Double space?

\hspace{24pt} Many of the traditional foodways of the peoples of the Dene First Nation are at risk of endangerment due to the multiplicity of factors associated with anthropogenic activity in the Yellowknife region, including deforestation, mining, and climate change \parencite{denefoodwaysontologies}.
It has been reported that traditional hunting and trapping grounds have been lost to human settlement and industry, and that fisheries, drinking water sources, and wild berry and medicine harvesting activities have had to relocate due to concerns over contamination from industrial activities \parencite{riskperceptions}.
On contamination, Jonas Sangris stated during the same interview that ``it's really bad~\ldots~they left what they call arsenic~\ldots~and they put it underground, and it's still there till today~\ldots~[arsenic] will all go in the water~\ldots~there's a lot of sickness on arsenic'' [37:46] \parencite{jonassangris}.

\section{Methodology}

\subsection{Research Question}

\hspace{24pt} The purpose of this research is threefold:
to explore the potential for the synthesis of Western agricultural practices and technologies with traditional Indigenous foodways in addressing food insecurity;
to examine and evaluate existing cases of this synthesis as they apply to Indigenous food sovereignty in the age of anthropogenic climate change and endangering of traditional foodways, and;
to propose how this learning might address food insecurity for the YKDFN through translocal knowledge exchange.

\hspace{24pt} This research will be primarily accomplished through literature review of existing studies on the subjects of food security, Indigenous food sovereignty, and the application of Western science to Indigenous foodways - as they pertain to the YKDFN and surrounding region.
The Western scientific aspect will be primarily focussed on two areas of food production, animal husbandry and greenhouse/indoor gardening, and will explore how these technologies might be adaptable to the traditional foodways of the YKDFN, and what practical limitations might exist.
The Indigenous aspect will be focussed on the following pillars of Indigenous food sovereignty: healthiness and cultural appropriateness of foods and food sources for individuals and the community, ecologically sound and sustainable food production, and localized and independent food systems.

\subsection{Thesis}

\hspace{24pt} It is argued herein that the synthesis of Western technologies with Indigenous food sovereignty practices can play a role in addressing food insecurity the territory of the YKDFN in the age of anthropogenic climate change and endangered traditional foodways, but that this role is limited by social, economic, and environmental factors. 
This thesis will be explored through the evaluation of existing integrations of Western technologies into Indigenous food sovereignty initiatives and through examination of the potential for translocal knowledge exchange in addressing food insecurity in the YKDFN while preserving traditional foodways.

\section{Reindeer Herding in Scandinavia}

\hspace{24pt} Reindeer and caribou (both \textit{Rangifer tarandus}) have been harvested by many groups of Indigenous Peoples in the circumpolar region for thousands of years as a source of food, clothing, and tools; for example, the herding of domesticated reindeer plays an essential role in the food system of the Indigenous S\'ami peoples of Scandinavia \parencite{traditionalsiberia}, and ``caribou continue to provide [the] primary environmental relationship'' \parencite[225]{denefoodwaysontologies} of the Dene people of Canada's Northwest Territories. 
Jonas Sangris in the same interview stated that the YKDFN ``live on caribou all year, all year round; they used it for food, clothing, and everything'' [32:38].
On the migration of caribou in response to anthropogenic activity, he stated that ``we have to travel about six, seven hours to try to get caribou; before, it wasn't like that. About 30 years ago, there was a route~\ldots~you go there November first, you're guaranteed a caribou - now, there's nothing. So that's how bad it is[, and] it's getting worse'' [35:36]. 
When asked what happened to the caribou, he replied that ``there's no food, they've got to look for food~\ldots~they're not going to go back~\ldots~they move away~\ldots~I don't know where they went'' [40:12] \parencite{jonassangris}.

\hspace{24pt} If subsistence on caribou is to continue as a viable solution to local food insecurity, it must be approached with a holistic understanding of the relationship between the people of the YKDFN and the land.
Conservation efforts are already in place in Yellowknife and the surrounding area to safeguard traditional caribou habitats as Indigenous Protected and Conserved Areas \parencite{resilientcommunities}, but when asked about the future of caribou in the region, Sangris stated that the government says ``\,`we'll try to manage the caribou, we'll have a plan', [but] did the caribou know that? We don't raise them, we don't feed them, the Creator put them on the land for us'' [43:49] \parencite{jonassangris}.
The Ministry of Environment and Natural Resources ``suggested that hunting must be curtailed to prevent extinction'' and implemented a ``limited tag system for [the YKDFN] to 300 animals annually; not enough to maintain a traditional subsistence lifestyle'' \parencite[230]{denefoodwaysontologies}.
This is reflective of a Western understanding of conservation that does not take into account the spiritual and social aspects of the traditional relationship between the Indigenous Peoples and the land.

\hspace{24pt} By contrast, in Scandinavia, active human involvement employing Indigenous knowledge-based traditional pasture management practices have been used with success to maintain the health of reindeer herds and the surrounding tundra ecosystem \parencite{reindeerfoodsovereignty}.
The application of traditional knowledge of the the S\'ami people to reindeer herding has been challenged by changing climate and encroachments on grazing land by industry. Unpredictable weather and the loss of grazing land mean that lichen, the herd's primary food source, is becoming increasingly difficult to source \parencite{samishiftingstrategies}.
A S\'ami herder named Rickard L\"anta stated in an interview published in 2022 that ``here where the planned mine would be~\ldots~If we can't use this area, we must find a new place, and then there would be conflicts with other reindeer herders~\ldots~because there isn't enough pasture to go around'' [1:31] \parencite{samiherders}.
The YKDFN have faced similar challenges: ``Almost the whole territory burned~\ldots~so [the caribou are] looking for food, so they sort of moved out'' [35:23] \parencite{jonassangris}. Without a realiable localized food source, booth caribou herding and hunting is becoming increasingly impossible for both the S\'ami and the YKDFN.
Because of this lack of grazing-land availability, ``Supplementary feeding is now needed in winter to keep the reindeer alive [and] allows herders to further increase herd sizes'' \parencite[35]{reindeerfoodsovereignty}. This presents a partial solution, but does not take into consideration difficulties around sourcing feed and implementing a herding system in the Yellowknife region.

\section{Indoor \& Greenhouse Community Gardening}

\hspace{24pt} Community gardening is already a well-established practice in the Yellowknife area, and has been used to address food insecurity in the past.
However, ``garden yields [alone] do not satisfy a community's need for fresh and nutritious foods'' \parencite[85]{resilientcommunities}, instead suggesting that local food production serves as a supplement to imported foods and a pillar of food sovereignty.
Greenhouses and other indoor growing methods have been used to varying degrees of success in the Northwest Territories to address food insecurity locally; for example, ``To combat the short, cooler growing seasons, communities [in the North] are using an array of growing infrastructure, such as greenhouses, hydroponics, and indoor spaces, to lengthen and improve growth conditions'' \parencite[90]{resilientcommunities}.
Growing fresh foods locally has become a necessity, as ``Fresh, nutritious market foods often arrive [from the South] to communities spoiled or damaged and without great diversity'' \parencite[85]{resilientcommunities}.

\hspace{24pt} In addressing these limitations, modern advancements in food production technology must be considered. For instance, aeroponic mist-based systems - which use far less fresh water than even hydroponic systems - have been shown to yield up to seventy percent greater crops of potatoes than hydroponic methods in a greenhouse environment \parencite{aeroponicpotatoes}.
This presents a significant opportunity, as potatoes are a staple crop for the Indigenous Peoples of the region: in an interview conducted in 2021, Dene man and then-Mayor of Fort Providence Danny Beaulieu is quoted as saying that ``Potato was the big thing for you to eat with meat'' \parencite[91]{resilientcommunities}. 
Furthermore, the ability for aeroponic systems to be both employed in an indoor environment and to be used for repeated harvesting from a single batch of plants makes them an ideal candidate for addressing food insecurity on a year-round basis.

\hspace{24pt} Beyond supplying fresh food to local groceries, indoor gardens provide year-round community gathering spaces, facilitate the exchange of knowledge, and provide full-time employment for community members: ``gardens [have been] identified as closer and more accessible areas to foster land-relationships and support cultural practices~\ldots~a space in the indoor garden [is] dedicated to Elders who experience physical and financial difficulties procuring traditional foods and medicines'' \parencite[99]{resilientcommunities}. Nevertheless, limitations to implementation of indoor community gardens include high energy costs for heating and power and the need for specialized equipment and knowledge. ``To combat these challenges, communities often link garden projects with established institutions, like schools or health centres, to pool resources, reduce knowledge gaps, and inspire greater interest in gardening'' \parencite[90]{resilientcommunities}.

% TODO: environmental, economic, and policy barriers to local food production

\section{Conclusion}

\hspace{24pt} The applicability of Western technologies to the traditional foodways of the Yellowknife Dene First Nation is an incomplete solution, and does not address the spiritual and social aspects of culture that are integral to the foodways of the Indigenous people of the Dene First Nation.
``For the traditional Dene, ideology is a fundamental element of subsistence, as important as practical empirical knowledge and appropriate technology'' \parencite[64]{lorecapturingtraditional}.
It is feared that, should it come to it, the Indigenous people of Canada's north may be forced to choose between adequate food security and the preservation of ancestral traditions due to the effects of anthropogenic activity in the region.
This may have deleterious consequences, as ``many of the social problems facing Indigenous peoples~\ldots~can be linked to the social vacuum that was created when harvesting ceased as the major focus of life'' \parencite[269]{socialculturalcapital}.
While hunted and trapped meat remain important, animal scarcity and loss of habitat may necessitate a transition toward more sustainable and reliable food sources, such as those provided by indoor and greenhouse gardening, or the domestication of caribou.

\hspace{24pt} In the case of caribou, herd management in the Yellowknife region has been approached from a Western scientific perspective, without consideration for the downstream effects on the spiritual and social aspects of the traditional relationship between the Dene people and the caribou.
It is speculated that a more active approach to caribou herd management that incorporates the traditional knowledge of other circumpolar Indigenous Peoples, such as the S\'ami, may be more effective than government intervention in supporting the overall sustainability of caribou subsistence hunting for the YKDFN.

\hspace{24pt} Indoor and greenhouse community gardens, on the other hand, present a proven and reliable step toward addressing food insecurity in Canada's northern regions, and have been used to great effect in the past to supplement imported foods year-round with fresh and nutritious locally-grown produce.
This particular solution addresses the need for healthy and culturally-appropriate foods, while localizing food production in an ecologically sound and sustainable manner.
Limitations presented by high energy costs could potentially be resolved through the implementation of green energy solutions, and the need for specialized knowledge could be addressed through partnership with established institutions (i.e. academia, government, and industry) and the exchange of knowledge between community members facilitated by the indoor spaces within which the gardens are operated.

\hspace{24pt} It is hoped that, rather than choosing between food security and cultural preservation, the application of both Western greenhouse technologies and Indigenous herding-knowledge can be used in tandem to address food insecurity in the Yellowknife region while preserving traditional foodways.
This will require significant investment, both economically and socially, in order to address the financial barriers to implementation and to overcome the cultural barriers to the adoption of unfamiliar practices.

\clearpage

% References
\printbibliography[heading=bibintoc]

\end{document}